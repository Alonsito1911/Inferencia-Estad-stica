% Options for packages loaded elsewhere
\PassOptionsToPackage{unicode}{hyperref}
\PassOptionsToPackage{hyphens}{url}
\PassOptionsToPackage{dvipsnames,svgnames,x11names}{xcolor}
%
\documentclass[
]{article}
\title{Problemas de Análisis de la varianza}
\author{}
\date{\vspace{-2.5em}}

\usepackage{amsmath,amssymb}
\usepackage{lmodern}
\usepackage{iftex}
\ifPDFTeX
  \usepackage[T1]{fontenc}
  \usepackage[utf8]{inputenc}
  \usepackage{textcomp} % provide euro and other symbols
\else % if luatex or xetex
  \usepackage{unicode-math}
  \defaultfontfeatures{Scale=MatchLowercase}
  \defaultfontfeatures[\rmfamily]{Ligatures=TeX,Scale=1}
\fi
% Use upquote if available, for straight quotes in verbatim environments
\IfFileExists{upquote.sty}{\usepackage{upquote}}{}
\IfFileExists{microtype.sty}{% use microtype if available
  \usepackage[]{microtype}
  \UseMicrotypeSet[protrusion]{basicmath} % disable protrusion for tt fonts
}{}
\makeatletter
\@ifundefined{KOMAClassName}{% if non-KOMA class
  \IfFileExists{parskip.sty}{%
    \usepackage{parskip}
  }{% else
    \setlength{\parindent}{0pt}
    \setlength{\parskip}{6pt plus 2pt minus 1pt}}
}{% if KOMA class
  \KOMAoptions{parskip=half}}
\makeatother
\usepackage{xcolor}
\IfFileExists{xurl.sty}{\usepackage{xurl}}{} % add URL line breaks if available
\IfFileExists{bookmark.sty}{\usepackage{bookmark}}{\usepackage{hyperref}}
\hypersetup{
  pdftitle={Problemas de Análisis de la varianza},
  colorlinks=true,
  linkcolor={red},
  filecolor={Maroon},
  citecolor={blue},
  urlcolor={blue},
  pdfcreator={LaTeX via pandoc}}
\urlstyle{same} % disable monospaced font for URLs
\usepackage[left=2cm,right=2cm,top=2cm,bottom=2cm]{geometry}
\usepackage{graphicx}
\makeatletter
\def\maxwidth{\ifdim\Gin@nat@width>\linewidth\linewidth\else\Gin@nat@width\fi}
\def\maxheight{\ifdim\Gin@nat@height>\textheight\textheight\else\Gin@nat@height\fi}
\makeatother
% Scale images if necessary, so that they will not overflow the page
% margins by default, and it is still possible to overwrite the defaults
% using explicit options in \includegraphics[width, height, ...]{}
\setkeys{Gin}{width=\maxwidth,height=\maxheight,keepaspectratio}
% Set default figure placement to htbp
\makeatletter
\def\fps@figure{htbp}
\makeatother
\setlength{\emergencystretch}{3em} % prevent overfull lines
\providecommand{\tightlist}{%
  \setlength{\itemsep}{0pt}\setlength{\parskip}{0pt}}
\setcounter{secnumdepth}{5}
\renewcommand{\contentsname}{Contenidos}
\ifLuaTeX
  \usepackage{selnolig}  % disable illegal ligatures
\fi

\begin{document}
\maketitle

{
\hypersetup{linkcolor=blue}
\setcounter{tocdepth}{4}
\tableofcontents
}
\hypertarget{ejercicios-independencia-y-homogeneidad}{%
\section{Ejercicios independencia y
homogeneidad}\label{ejercicios-independencia-y-homogeneidad}}

\hypertarget{problema-1}{%
\subsection{Problema 1}\label{problema-1}}

Doce personas son distribuidas en \(4\) grupos de personas \(3\) cada
uno. A cada grupo, se le asigna aleatoriamente un tiempo distinto de
entrenamiento antes de realizar una tarea. Los resultados en la
mencionada tarea, con el correspondiente tiempo de entrenamiento, son
los siguientes:

\begin{center}
\begin{tabular}{|c|c|c|c|}
\hline
$0.5$ horas&$1$ hora&$1.5$ horas&$2$ horas\\\hline\hline
$1$&$4$&$3$&$\ \,8$\\\hline
$3$&$6$&$5$&$10$\\\hline
$5$&$2$&$7$&$\ \,6$\\\hline
\end{tabular}
\end{center}

Ver si podemos rechazar la hipótesis nula
\(H_0:\mu_1=\mu_2=\mu_3=\mu_4.\)

\hypertarget{problema-2}{%
\subsection{Problema 2}\label{problema-2}}

Se registraron las frecuencias de los días que llovió a diferentes
horas, durante los meses de enero, marzo, mayo y julio. Los datos
obtenidos, durante un periodo de 10 años, fueron los siguientes:
~\newline

\begin{center}
\begin{tabular}{|c||c|c|c|c||c|}
\hline
Hora&enero&febrero&marzo&julio&Total\\\hline\hline
$\ \,9$&$\ \,22$&$\ \,25$&$\ \,24$&$\ \,11$&$\ \,82$\\\hline
$10$&$\ \,21$&$\ \,19$&$\ \,18$&$\ \,16$&$\ \,74$\\\hline
$11$&$\ \,17$&$\ \,23$&$\ \,26$&$\ \,17$&$\ \,83$\\\hline
$12$&$\ \,20$&$\ \,31$&$\ \,25$&$\ \,24$&$100$\\\hline
$13$&$\ \,16$&$\ \,15$&$\ \,23$&$\ \,24$&$\ \,78$\\\hline
$14$&$\ \,21$&$\ \,35$&$\ \,23$&$\ \,20$&$\ \,99$\\\hline\hline
Total&$117$&$148$&$139$&$112$&$536$\\\hline
\end{tabular}
\end{center}

~\newline Estudiar la variabilidad entre meses y entre horas.

\hypertarget{problema-3}{%
\subsection{Problema 3}\label{problema-3}}

Se realizó un estudio para determinar el nivel de agua y el tipo de
planta sobre la longitud global del tronco de las plantas de guisantes.
Se utilizaron \(3\) niveles de agua y \(2\) tipos de plantas. Se dispone
para el estudio de \(18\) plantas sin hojas. Las plantas se dividen
aleatoriamente en \(3\) subgrupos y después se los asigna los niveles de
agua aleatoriamente. Se sigue un procedimiento parecido con \(18\)
plantas convencionales. Se obtuvieron los resultados siguientes (la
longitud del tronco se da en centímetros): ~\newline

\begin{center}
\begin{tabular}{c|c|c|c|c|}
&&\multicolumn{3}{c|}{FACTOR AGUA}\\\hline
& &{bajo}&{medio}&{alto}\\\cline{3-5}
\multirow{12}{1.75cm}{FACTOR PLANTA}&\multirow{6}{1cm}{Sin Hojas}&
$69.0$&$\ \,96.1$&$121.0$\\\cline{3-5}
&&$71.3$&$102.3$&$122.9$\\\cline{3-5}
&&$73.2$&$107.5$&$123.1$\\\cline{3-5}
&&$75.1$&$103.6$&$125.7$\\\cline{3-5}
&&$74.4$&$100.7$&$125.2$\\\cline{3-5}
&&$75.0$&$101.8$&$120.1$\\\cline{2-5}
&\multirow{6}{1cm}{Con Hojas}&$71.1$&$\ \,81.0$&$101.1$\\\cline{3-5}
&&$69.2$&$\ \,85.8$&$103.2$\\\cline{3-5}
&&$70.4$&$\ \,86.0$&$106.1$\\\cline{3-5}
&&$73.2$&$\ \,87.5$&$109.7$\\\cline{3-5}
&&$71.2$&$\ \,88.1$&$109.0$\\\cline{3-5}
&&$70.9$&$\ \,87.6$&$106.9$\\\hline
\end{tabular}
\end{center}

~\newline Se desea saber si hay diferencias entre los niveles de agua y
entre los diferentes tipos de planta. También se quiere saber si hay
interacción entre los niveles de agua y el tipo de planta.

\hypertarget{problema-4}{%
\subsection{Problema 4}\label{problema-4}}

Las variables aleatorias \(X_i\) siguen la distribución
\(N(m_i,\sigma^2),\ i=1,2,3,4\). Consideramos las siguientes muestras de
tamaños \(n_i=7\) de las mencionadas variables aleatorias: ~\newline

\begin{center}
\begin{tabular}{cccccccc}
$X_1$&$20$&$26$&$26$&$24$&$23$&$26$&$21$\\
$X_2$&$24$&$22$&$20$&$21$&$21$&$22$&$20$\\
$X_3$&$16$&$18$&$20$&$21$&$24$&$15$&$17$\\
$X_4$&$19$&$15$&$13$&$16$&$12$&$11$&$14$\\
\end{tabular}
\end{center}

~\newline a) Comprobar si las varianzas son iguales. b) Contrastar la
igualdad de medias.

\end{document}
